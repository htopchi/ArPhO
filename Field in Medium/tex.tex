\addcontentsline{toc}{section}{Field in Medium (5)}
\section*{Field in Medium}

\subsection*{Problem}

\renewcommand{\labelenumii}{\theenumii}
\renewcommand{\theenumii}{\theenumi.\arabic{enumii}.}

\begin{enumerate}

{\bfseries
\item Dielectrics
}

\begin{enumerate}
    \item What dipole moment will a conducting ball of radius $a$ gain
        in an electric field of strength $E$?

    \item What is the strength of electric field inside a capacitor
        with dipole moment density $p$?

    \item What is the electric permittivity of a medium
        with concentration of atoms $n$,
        if we consider the atoms as conducting ball of radius $a$.
\end{enumerate}

{\bfseries
\item Ferromagnetics
}

It is know
that atoms have a magnetic moment,
which can be explained by
the rotation of the electron around the nucleus (but it's not).
According to quantum theory,
in case of existence of an external magnetic field
that magnetic moment will either align with it
or take the direction opposite to it.
Furthermore,
the probability of each of these states is given by
Boltzmann distribution, i.e.
\begin{equation}
    \varphi \propto \exp(-\frac{E}{k_B T})
\end{equation}
where $\varphi$ is the probability of the state,
$E$ is its energy, $k_B$ is Boltzmann constant
and $T$ is the absolute temperature.


\begin{enumerate}
    \item Find the magnetic moment expected value of 
        a magnetic dipole with moment $\mu$
        inside a uniform magnetic field $\vec{B}$.

    \item What is the strength of magnetic field inside a coil
        with magnetic moment density $\vec{\kappa}$?

    \item What is the magnetic permeability of a medium
        with concentration of atoms $n$ which have magnetic moment $\mu$,
        if we consider the field generated by the medium itself
        to be much smaller than the external field?
\end{enumerate}

It turns out that ferromagnetics have a domain structure.
This means that there are regions (domains) if atoms
with same magnetic moment direction
rather then each atom being independent.
As a result the magnetic field at any given point
is considerably stronger than the mean magnetic field.

\hspace{.5cm}
To encapsulate all the effects of the domain structure,
suppose that there is some effective magnetic field
$B_{eff} = B + b I$ acting at each point in direction of $\vec{B}$,
where $B$ is the external field,
$I$ is the mean field generated by the medium
and $b > 1$ is some parameter describing the matter.

\begin{enumerate}
    \setcounter{enumii}{3}
    \item Write down an equation for $I$
        (an equation, where only $I$ is unknown).

    \item At what temperatures it is possible for a matter with parameters
        $n$, $b$ and $\mu$ to be permanent magnet?
        The critical temperature is called Curie temperature.

    \item Calculate the numerical value of $b$ for iron, if known that
    $k_B = 1.4 \cdot 10^{-23} \text{ J/K}$,
    $\mu_0 = 1.3 \cdot 10^{-6} \text{ N/A$^2$}$,
    $\mu = 9.3 \cdot 10^{-24} \text{ J/T}$,
    $m_{\text{proton}} = 1.7 \cdot 10^{-27} \text{ kg}$,
    $M_{Fe} = 56 \text{ g/mol}$,
    $\rho_{Fe} = 7.9 \cdot 10^3 \text{ kg/m$^3$}$,
    $T_{Fe}^C = 1.0 \cdot 10^3 \text{ K}$.
\end{enumerate}

\end{enumerate}

\subsection*{Useful information}
\begin{itemize}
    \item Energy of a magnetic dipole $\vec{\mu}$ in magnetic field $\vec{B}$
        is given by $E = -\vec{\mu} \cdot \vec{B}$.

    \item Magnetic moment of a flat contour with current $I$ and area $S$ is
        $\vec{\mu} = I\vec{S}$.

    \item The solution of an equation is the intersection point of graphs
        of expressions on the two sides of the equation.
\end{itemize}

\subsection*{Solution}

\begin{enumerate}

{\bfseries
\item Dielectrics
}

\begin{enumerate}
    \item Dipole moment of a ball in external field
\end{enumerate}

\hspace{.5cm}
This is a famous problem.
The only reason of including it here is that the result is needed later.
If we consider two uniformly charged balls of radius $a$ and charges $\pm q$,
with displacement $\vec{d} \ll a$,
the result will be a non-uniformly charged sphere of radius $a$.

\hspace{.5cm}
To calculate the field inside this object
we have to sum up the fields of the two balls.
It is known, that at some distance $r$ form the center
only the charge "inside" that distance crates field.
Suppose the ball with charge $-q$ is at $0-0$
and the other one is at $\vec{d}$.
The dipole moment is $\vec{P} = q \vec{d}$.
Their field is then given by
\begin{equation}
    E(\vec{r}) = \frac{-k q}{r^2} \cdot \frac{r^3}{a^3} \cdot \frac{\vec{r}}{r} +
        \frac{k q}{r_d^2} \cdot \frac{r_d^3}{a^3} \cdot \frac{\vec{r_d}}{r_d} =
        -\frac{kq}{a^3}(\vec{r} - \vec{r}_d) = -\frac{k \vec{P}}{a^3}
\end{equation}
where $\vec{r}_d = \vec{r} - \vec{d}$.

\hspace{.5cm}
As the field inside is uniform,
once inside an external electric field,
the ball will charge in way we just discussed
to compensate the external field inside of it
(the field in a conductor should be $0$).
This will be achieved if the ball gains dipole moment
$\vec{P} = \vec{E} a^3 / k$.

\begin{enumerate}
    \setcounter{enumii}{1}
    \item Electric field in a capacitor
\end{enumerate}

\hspace{.5cm}
Consider a capacitor of area $S$ and thickness $d$,
that is charged with surface charge density $\pm \sigma$.
The overall dipole moment in this case is $P = \sigma S \cdot d$
(in direction towards the plate charged with $\sigma$)
and its density is thereby $\vec{p} = \sigma$.

\hspace{.5cm}
The electric field inside can be given by $E = \sigma / \varepsilon_0$
towards the plate with $-\sigma$.
So the field can be expressed in terms of $\vec{p}$ as
\begin{equation}    
    \vec{E} = -\vec{p} / \varepsilon_0
    \labele{E_on_p}
\end{equation}

\begin{enumerate}
    \setcounter{enumii}{2}
    \item Electric permittivity of a medium
\end{enumerate}

\hspace{.5cm}
Electric permittivity is the ratio of the initial external field
to what we have in the medium as a result of superposition with the medium's field.
The source of medium's field is the polarization of atoms
under the influence of both external and medium-generated fields.
If we denote the later by $\vec{I}$,
the polarization density of the medium is then
\begin{equation}
    \vec{p} = \frac{n \inb({\vec{E} + \vec{I}}) a^3}{k}
    \labele{p_on_EI}
\end{equation}
We already know the strength of the field that this will create.
By plugging \refe{p_on_EI} into \refe{E_on_p} we get
\begin{equation}
    \vec{I} = -\frac{n \inb({\vec{E} + \vec{I}}) a^3}{k \varepsilon_0}=
        -4\pi n \inb({\vec{E} + \vec{I}}) a^3
\end{equation} 
From here we can get $\vec{I}$ and
the permittivity is given by
$\varepsilon = \vec{E} / \inb({\vec{E} + \vec{I}}) = 1 + 4 \pi n a^3$

{\bfseries
\item Ferromagnetics
}

\begin{enumerate}
    \item Magnetic moment expected value
\end{enumerate}

\hspace{.5cm}
Let's denote $x = \mu B / k_B T$.
Also let's have axis $z$ in direction of $\vec{B}$.
The probability of $\vec{\mu}$ being along $\vec{B}$
is $\varphi_{\uparrow} = a e^x$, and for being opposite to $\vec{B}$
it is $\varphi_{\downarrow} = a e^{-x}$.
The coefficient $a$ can be found using the fact
that these are the only possible states.
For expected value of $\mu$ we then have
\begin{equation}
    \inb<{\mu_z}> = \mu \varphi_{\uparrow} - \mu \varphi_{\downarrow} = 
    \mu \frac{e^x - e^{-x}}{e^x + e^{-x}} = \mu \tanh{x}
\end{equation}

\begin{enumerate}
    \setcounter{enumii}{1}
    \item Magnetic field in a coil
\end{enumerate}

\hspace{.5cm}
Consider a coil with section area $S$, length $L$
and winding distance $d$, with current $I$ in it.
The current flows up and counterclockwise when looking from above.
The magnetic field inside it is given by $B = \mu_0 I / d$,
pointing up.
The magnetic moment of a single winding is $IS$, also pointing up.
The direction can be understood by supposing a square contour
that is not perpendicular to some magnetic field,
and observing the forces that are acting on it,
which we know, want to alight the moment with the field,
as that state has the lowest energy.

\hspace{.5cm}
The magnetic moment density is then $\kappa = I / d$.
Finally $\vec{B}$ in terms of $\vec{\kappa}$ is given as
$\vec{B} = \mu_0 \vec{\kappa}$.

\begin{enumerate}
    \setcounter{enumii}{2}
    \item Magnetic permeability of a medium
\end{enumerate}

\hspace{.5cm}
Magnetic permeability is the ratio of the field inside the medium
to the initial magnetic field.
The source of medium's field is the polarization of atoms
under the external fields (as we can neglect the medium's field here).
Notice that unlike the dielectrics,
the medium generates a field along the external filed and not opposite to it.
If we denote the internally created field by $I$,
then the magnetic permeability is
\begin{equation}
    \mu^m = \frac{\vec{B} + \vec{I}}{\vec{B}} =
    1 + \frac{\mu_0 n \mu \tanh{x}}{B} \approx
    1 + \frac{\mu_0 n \mu x}{B} =
    1 + \frac{n \mu_0 \mu^2}{k_B T}
\end{equation}

\begin{enumerate}
    \setcounter{enumii}{3}
    \item Strong internal field
\end{enumerate}

\hspace{.5cm}
According to the model,
we should consider the action of the effective field
on each individual atom,
which as a result will generate the internal field.
The corresponding equation is
\begin{equation}
    I = n \mu_0 \mu \tanh(\frac{\mu (B + bI)}{k_B T})
    \labele{I_on_BI}
\end{equation}

\begin{enumerate}
    \setcounter{enumii}{4}
    \item Permanent magnet
\end{enumerate}

\hspace{.5cm}
The term "permanent magnet" implies
it can sustain its own magnetic field without any external field.
So the problem reduces to determining the temperatures,
for which \refe{I_on_BI} has a solution with $B = 0$.

\hspace{.5cm}
On the left-hand side of the equation is just $I$,
and on the right-hind side it's an $I$ in $\tanh()$.
Both these functions are $0$ at $I=0$.
Moreover, $\tanh(x)$ has a decreasing derivative,
which eventually reaches $0$.
This means, that the equation has a solution other than $0$
if and only if the derivative of right-hand side at $0$ is greater than $1$.
Formally it is written as
\begin{equation}
    \inb.{n \mu_0 \mu \derivative{}{I} \tanh(\frac{\mu bI}{k_B T})}|_{I=0} =
    \frac{n \mu_0 \mu^2 b}{k_B T} \inb.{\cosh^{-2} \inb({\frac{\mu bI}{k_B T}})}|_{I=0} =
    \frac{n \mu_0 \mu^2 b}{k_B T} > 1
\end{equation}
So the temperature should be $T < T^C = n \mu_0 \mu^2 b / k_B$.

\begin{enumerate}
    \setcounter{enumii}{5}
    \item $b$ of iron
\end{enumerate}

\hspace{.5cm}
The only parameter not known to calculate $b$ is the concentration,
which can be easily found using the molar mass, density and the mass of proton
as $n_{Fe} = \rho_{Fe} / m_p [M_{Fe} \text{ mol/g}]$,
where the square braces emphasize that we take a dimensionless parameter,
namely $56$.

\hspace{.5cm}
Then for $b$ we get $b_{Fe} = k_B T^C_{Fe} / \mu_0 \mu^2 n_{Fe} \approx 1500$.

\end{enumerate}
