\addcontentsline{toc}{section}{Rolling in place (3)}
\section*{Rolling in place}

\subsection*{Problem}

Suppose a big horizontal disk,
rotating around its center with angular velocity $\Omega$.
It is possible to place a solid ball at any point of the disk
in such a way, that the ball doesn't move.
What will happen, if one gives velocity $\vec{v}_0$ to the ball?
What changes if we substitute the ball with a sphere?
There is no sliding.

\subsection*{Useful information}
\begin{itemize}
    \item The angular velocity of the ball can be represented as
        a vector in the plane of the disk.
\end{itemize}

\subsection*{Solution}

Let's use notation $\vec{\omega}$ for ball's angular velocity,
which is a vector direction where the ball would roll
if it was on a static surface.

Initially the ball is at some point $\vec{r}_0$ and it doesn't move,
which means there is no friction force acting on it.
It can be implied, that the velocity of the bottom point of it
is the same as the speed of the disk's surface $\vec{\Omega} \times \vec{r}_0$.
The ball's bottom point's velocity is given by $-\vec{\omega} R$,
where $R$ is the radius of the ball.

Once given velocity, the ball will start rolling around,
but at any point $\vec{r}$ the condition
$\vec{\Omega} \times \vec{r} = \vec{v} - \vec{\omega} R$
should be satisfied,
where $\vec{v}$ is the velocity of the ball's center.
The equation is just the equality of the ball's bottom point's velocity
to that of the disk at that specific point.
Another thing that we know about this system,
is that both $\vec{\omega}$ and $\vec{v}$ change due to
the same force, which is the friction force $\vec{F}$.
If $m$ is the mass of the ball, then
\begin{equation}
    \frac{2 m R^2}{5} \cdot \derivative{\vec{\omega}}{t} = -\vec{F} R
    \hspace{1cm} m \derivative{\vec{v}}{t} = \vec{F}
    \labele{mot_eq}
\end{equation}
which immediately produces
$\differential{\vec{\omega}} R = -5 \differential{\vec{v}} / 2$.
We can put this into the derivative of no-sliding equation.
The result will be
\begin{equation}
    \vec{\Omega} \times \vec{v} \differential{t} =
    \frac{7}{2} \differential{\vec{v}}
\end{equation}
So $\differential{\vec{v}}$-'s direction is to the left of $\vec{v}$,
perpendicular to it.
This suggests, that the ball's velocity module doesn't change,
and the ball is doing a counterclockwise circular motion
with angular velocity $2 \Omega / 7$.
The radius of the circle is $7 v_0 / 2 \Omega$.

The only change in case of substituting the ball with a sphere,
is the change of factor $2/5$ in \refe{mot_eq} to $2/3$,
which leads to alternative angular velocity of circular motion $2 \Omega / 5$
and corresponding radius $5 v_0 / 2 \Omega$.

