\addcontentsline{toc}{section}{Thick Barrel (4)}
\section*{Thick Barrel (4)}

\subsection*{Problem}
What thickness should a cylindrical barrel with internal radius $R$ have,
in order to withstand internal pressure $P$?
The barrel is made of material with strength $\sigma$.
What happens in case $P \ll \sigma$?

\subsection*{Useful information}
\begin{itemize}
    \item The solutions to differential equation $y'' = y / x^2$
        obviously have form $y = C x^a$.
    \item When $k>1$
        \begin{equation*}
            \inb({\frac{2}{5 + \sqrt{5}} + \frac{2}{5 - \sqrt{5}} k^{\sqrt{5}}}) > k^{\frac{\sqrt{5} - 1}{2}}
        \end{equation*}
\end{itemize}

\subsection*{Solution}

We will be describing the state of our barrel
with the radial displacement of each point $x(r)$,
$r \rightarrow r + x(r)$.
The tangential component of relative deformation then will be
\begin{equation}
    \varepsilon_t(r) = \frac{x(r)}{r}
\end{equation}
as the ring with initial length $2\pi r$ now has length $2\pi (r + x(r))$.
The radial deformation is given by
\begin{equation}
    \varepsilon_r(r) = \frac{\differential x(r)}{\differential r}
\end{equation}
as the point at initial positions $r$ and $r + \Delta r$ are now at
$r + x(r)$ and $r + \Delta r + x(r + \Delta r) = r + \Delta r + x(r) + x'(r) \Delta r$.
So the point now have distance $(1 + x'(r)) \Delta r$ instead of initial $\Delta r$.

The internal (inside the barrel walls) equilibrium is generated
as a result of opposition of the radial forces to tangential forces.
Consider a section of a barrel at radius $r$,
with thickness $\differential h$, length (along the axis) $l$
and angular size $\differential \alpha$.
The equilibrium condition of it is written as
\begin{equation}
    2 E \varepsilon_t \cdot l \differential h \cdot \frac{\differential \alpha}{2} =
    E \frac{\differential \varepsilon_r}{\differential r} \differential h \cdot l r \differential \alpha
\end{equation}
where the left-hand side of the equation are the tangential pressure forces
projected on radial direction,
and the right-hand side is the difference between the radial pressure forces
from above and below (in terms of radii) of the section.

In terms of $x(r)$, this equation can be rewritten as simply $x''=x/r^2$.
The solutions are easily found in form $Cx^a$,,
from where we get the general solution of the equation
\begin{equation}
    x = C_+ r^{\beta_+} + C_- r^{\beta_-}
    \hspace{1cm}
    \beta_\pm = \frac{1 \pm \sqrt{5}}{2}
\end{equation}

The constants are determined by edge conditions
$E \varepsilon_r(R) = -P$ and $E \varepsilon_r(R_o) = 0$,
where $R_o$ is the outer radius of the barrel.
To apply these conditions we need the expression for $\varepsilon_r$.
\begin{equation}
    \varepsilon_r(r) = x'(r) = 
    C_+ \beta_+ r^{\beta_+ - 1} +
    C_- \beta_- r^{\beta_- - 1}
\end{equation}
The edge conditions result in
\begin{equation}
    \beta_+ C_+ = \frac{P}{E R^{\beta_+ - 1}
    \inb({ \inb({\frac{R_o}{R}})^{\sqrt{5}} - 1}) } 
    \hspace{1cm} \beta_- C_- = -\beta_+ C_+ R_o^{\sqrt{5}}
\end{equation}
where $\sqrt{5}$ comes from $\beta_+ - \beta_-$.
Note that $C_{\pm} > 0$.

Now we have to figure out where and in which direction
the barrel will break.
To this end we have to know where and in which directions
the existing tension (which is proportional to $\varepsilon$-s) is the highest.

$\varepsilon_r$ is a sum of increasing functions,
as $C_+ \beta_+ > 0$ , $\beta_+ - 1 > 0$ and
$C_- \beta_- < 0$ , $\beta_- - 1 < 0$.
Given that at $r = R_o$ it is $0$,
it has its highest absolute value at $r = R$.

The expression for $\varepsilon_t$ is just $\varepsilon_t = C_+ r^{\beta_+ -1} + C_- r^{\beta_- -1}$.
As a sum of increasing and decreasing positive monomials ($C_\pm > 0$ , $\beta_+ -1 > 0$, $\beta_- -1 < 0$),
$\varepsilon_t$ is a function with a single minimum,
which means that for any given interval $[a, b]$,
the maximum value of the function is either at $a$ or at $b$.

So the candidates of the highest tension are
\begin{equation}
\begin{split}
    |\varepsilon_r(R)| &= P/E \\
    \varepsilon_t(R) &= \frac{P}{E \inb({ k - 1})} \inb({\frac{1}{\beta_+} + \frac{k}{\beta_-}}) \\
    \varepsilon_t(R_o) &= \frac{P \sqrt{5}}{E \inb({ k - 1})} \inb({\frac{R_o}{R}})^{\beta_-}
\end{split}
\hspace{1cm} k = \inb({\frac{R_o}{R}})^{\sqrt{5}}
\end{equation}
The last two expressions seem remotely similar to terms in
inequation in the hint.
After some transformations one can show the connection,
with $k$ in the hint being $R_o / R$.
A a result we learn that $\varepsilon_t(R) > \varepsilon_t(R_o)$.

It is left to compare $|\varepsilon_r(R)|$ and $\varepsilon_t(R)$,
that is
\begin{equation}
    1 * \frac{\beta_- + \beta_+ k}{(k - 1) \beta_+ \beta_-}
\end{equation}
It is straightforward to show that the right-hand side is bigger.

Finally we have to demand $E \varepsilon_t(R) < \sigma$
for the barrel to remain integrate.
This leads to a linear equation on $k$ which results in
\begin{equation}
    k = \frac{1+\frac{P}{\sigma \beta_+}}{1-\frac{P}{\sigma \beta_-}}
\end{equation}
The corresponding thickness is $h = R (k^{1/\sqrt{5}} - 1)$.

For very small values of pressure the expression can be decomposed
\begin{equation}
    h_{\ll} = R \inb({ 1 + \frac{P}{\sqrt{5} \sigma \beta_+}
    + \frac{P}{\sqrt{5} \sigma \beta_-} - 1}) = \frac{RP}{\sigma}
\end{equation}
