\addcontentsline{toc}{section}{Heating Rod (4)}
\section*{Heating Rod}

\subsection*{Problem}
In this problem we will be considering a long cylindrical rod of radius $r$,
which is made of material with temperature dependent resistivity
$\rho = \rho_0 (1+\alpha T)$ (temperature scale is not necessarily absolute).
The material's density is $\mu$, heat conductivity is $\kappa$,
heat capacity is $c$, heat transfer coefficient with air is $\beta$.
Air temperature is $\tau$.
The rod is under voltage given by its average per unit length: $\inb<E>$.
Radial inhomogeneity of temperature in the rod can be neglected.
Use natation $\inb<\bullet>$ for average over space coordinate.

\begin{enumerate}
    \item Write the heat flow equation in the rod.
    Determine the uniform equilibrium temperature $T_0$.
    \item Consider a small non-uniform fluctuation $\delta T$ form equilibrium.
    In which cases the system will be stable against local fluctuations?
    Use $T_0$ without substitution for its expression.
\end{enumerate}

\subsection*{Useful information}
\begin{itemize}
    \item When looking for a solution of differential equation
    for function of two arguments $f(x,t)$,
    it can be useful to split it into linear combination of
    basic functions $b_\lambda(x)$ with time dependent coefficients
    $a_\lambda(t)$.
    In this case taking sines and cosines as basic functions
    (Fourier basis) proves useful.
\end{itemize}

\subsection*{Solution}

The heat equation has three contributions: current generated heat,
heat transfer to air and heat flow in the rod.
In case of given $T(x)$, the current in the rod will be given by
\begin{equation}
    I = \frac{U}{R} = \frac{\inb<E>l}{\int_0^l
    \rho_0 (1+\alpha T) \differential x / S} =
    \frac{\inb<E>S}{\rho_0(1+\alpha \inb<T>)}
\end{equation}
where $l$ is the length of the rod, and $S$ is its cross section.
The corresponding generated power at interval $\differential x$ is then
\begin{equation}
    \differential W_I = I^2 \differential R =
    \inb({\frac{\inb<E>S}{\rho_0(1+\alpha \inb<T>)}})^2
    \frac{\rho_0(1+\alpha T)\differential x}{S}=
    \frac{\inb<E>^2 S}{\rho_0} \cdot
    \frac{1+\alpha T}{(1+\alpha \inb<T>)^2} \cdot \differential x
\end{equation}
The power of heat exchange due to air and as a result of conductivity are
\begin{equation}
    \differential W_A = -\beta (T-\tau) \cdot 2\pi r \differential x 
    \quad , \quad 
    \differential W_C = \kappa S \frac{\differential^2 T}{\differential x^2} \differential x
\end{equation}
correspondingly. The overall equation is then
\begin{equation}
    \frac{\differential T}{\differential t} =
    \frac{\differential W_I + \differential W_A + \differential W_A}
    {\mu c S \differential x}=\frac{1}{\mu c}\inb({
        \frac{\inb<E>^2}{\rho_0} \cdot
        \frac{1+\alpha T}{(1+\alpha \inb<T>)^2} -
        \frac{2\beta (T-\tau)}{r} +
        \kappa \frac{\differential^2 T}{\differential x^2}
    })
\end{equation}

The condition of uniform equilibrium temperature is given by substitution
$T(x)=\inb<T>=T_0=\text{const}$.
\begin{equation}
    \frac{\inb<E>^2}{\rho_0} \cdot
    \frac{1}{1+\alpha T_0} -
    \frac{2\beta (T_0-\tau)}{r} = 0 \Rightarrow
    T_0 = \frac{-(1-\alpha \tau) \pm \sqrt{(1+\alpha \tau)^2 +
    \frac{2 \alpha r \inb<E>^2}{\rho_0 \beta}}}{2 \alpha}
    \labele{equil}
\end{equation}
The '$+$' sign must be chosen, because that's the one which produces
$T_0 = \tau$ in case $\inb<E>=0$.

Now let's consider a fluctuation $T=T_0+\delta T$.
The heat flow equation becomes
\begin{equation}
    \frac{\differential \delta T}{\differential t} = \frac{1}{\mu c}
    \inb({
        \frac{\inb<E>^2}{\rho_0} \cdot
        \frac{1+\alpha T_0+\alpha\delta T}
        {(1+\alpha T_0+\alpha \inb<{\delta T}>)^2} -
        \frac{2\beta (T_0+\delta T-\tau)}{r} +
        \kappa \frac{\differential^2 \delta T}{\differential x^2}
    })
    \labele{d_dev}
\end{equation}
as $T_0$ doesn't depend neither on $x$ nor on $t$.
The first term can be expanded into Taylor series
by small parameter $\delta T$.
\begin{equation}
    \frac{1+\alpha T_0+\alpha\delta T}
        {(1+\alpha T_0+\alpha \inb<{\delta T}>)^2}=
    \frac{1}{1+\alpha T_0}+\frac{\alpha \delta T}{(1+\alpha T_0)^2}
    -\frac{2\alpha \inb<{\delta T}>}{(1+\alpha T_0)^2}
\end{equation}

Afterwards we can subtract \refe{equil} from \refe{d_dev} and get
\begin{equation}
    \frac{\differential \delta T}{\differential t} =\frac{1}{\mu c}
    \inb({
        \frac{\alpha \inb<E>^2}{\rho_0 (1+\alpha T_0)^2} \cdot
        (\delta T - 2\inb<{\delta T}>) -
        \frac{2\beta \delta T}{r} +
        \kappa \frac{\differential^2 \delta T}{\differential x^2}
    })
\end{equation}
For small local fluctuations on a long rod $\inb<{\delta T}>$ can be neglected,
and we are left with
\begin{equation}
    \frac{\differential \delta T}{\differential t} =\frac{1}{\mu c}
    \inb({
        \inb({\frac{\alpha \inb<E>^2}{\rho_0 (1+\alpha T_0)^2}
        -\frac{2\beta}{r}}) \delta T +
        \kappa \frac{\differential^2 \delta T}{\differential x^2}
    })\equiv A \delta T + B \frac{\differential^2 \delta T}{\differential x^2}
\end{equation}

Let's use the hint, and look for $\delta T$ decomposition
$\sum_\lambda a_\lambda(t) \sin \lambda x$
(the same will be true for $\cos \lambda x$).
If looking at a component with specific $\lambda$, the equation reduces to
\begin{equation}
    \frac{\differential a_\lambda}{\differential t} =
    A a_\lambda - B \lambda^2 a_\lambda
\end{equation}
We want all possible fluctuations to decay,
which is provided by condition $A - B \lambda^2 < 0$ for any $\lambda$,
implying a need for negative $A$.

As a result the stability condition is given by
\begin{equation}
    \frac{\alpha \inb<E>^2}{\rho_0 (1+\alpha T_0)^2} < \frac{2\beta}{r}
\end{equation}

One can notice, that the two sides are exactly equal
in case $\inb<E> \gg \sqrt{\rho_0 \beta/\alpha r}$,
seemingly implying preservation of $\lambda=0$ (uniform) fluctuation in first order.
But it's not a local fluctuation,
so the term $\inb<{\delta T}>$ can't be neglected in this case.
Anyway... it can be checked,
that the the left-hand side is an increasing function on $\inb<E>$,
which means that stability condition is always satisfied.
