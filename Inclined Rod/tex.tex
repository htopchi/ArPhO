\addcontentsline{toc}{section}{Inclined Rod (2)}
\section*{Inclined Rod}

\subsection*{Problem}

A rod is hanged form the ceiling by its ends with
strings of length $l_1$ and $l_2$ in such a way,
that both strings are vertical.
Find the natural frequencies of rods oscillations. 

\subsection*{Solution}

There are three modes of oscillation.
The first one is the oscillation within the strings' plane,
and the other two include oscillations perpendicular to it
alongside with rotational oscillations.

Consider the first mode.
Let's denote the small inclinations of the strings
with length $l_1$ and $l_2$ by $\alpha$ and $\beta$ correspondingly.
The heights of the ends of the rod don't change,
so the constancy of rod length is written as $l_1 \alpha = l_2 \beta$.
We will be doing substitutions $\sin{\alpha} \rightarrow \alpha$
and $\cos{\alpha} \rightarrow 1$ without mentioning it.
There is also no rotation, so $T_1 = T_2$ (the string tension forces).

If the displacement of the rod is $x$,
then $\alpha = x / l_1$ and $\beta = x / l_2$.
The overall horizontal force will be
\begin{equation}
    F = -T_1 \sin{\alpha} - T_2 \sin{\beta} =
    \frac{mg}{2}\inb({\frac{1}{l_1} + \frac{1}{l_2}}) x
\end{equation}
As $F = m \ddot{x}$, 
the corresponding frequency will be
\begin{equation}
    \omega_1 = \sqrt{\frac{g}{2} \inb({\frac{1}{l_1} + \frac{1}{l_2}})}
\end{equation}

For the other two modes let's use the same $\alpha$ and $\beta$,
but here they will be showing the inclinations perpendicular ot strings' plane.
The displacements of rod's edges will be $l_1 \alpha$ and $l_2 \beta$.
The displacement $x$ of rod's center
and rod's rotation $\gamma$ around the vertical axis are then give by
\begin{equation}
\begin{split}
    x &= \frac{l_1 \alpha + l_2 \beta}{2} \\
    \gamma &= \frac{l_1 \alpha - l_2 \beta}{d}
\end{split}
\hspace{1cm} \implies \hspace{1cm}
\begin{split}
    \alpha &= \frac{2 x + \gamma d}{2 l_1} \\
    \beta &= \frac{2 x - \gamma d}{2 l_2}
\end{split}
\end{equation}
where $d$ is the horizontal distance between the strings.
The string tensions are also equal to $mg / 2$ in this case.
The motion equations are then written as
\begin{equation}
\begin{split}
    m \ddot{x} = -T_1 \sin{\alpha} - T_2 \sin{\beta} &=
    -\frac{mg}{2} (\alpha + \beta) =
    -\frac{mx}{2} \Sigma - \frac{md\gamma}{4} \Delta \\
    I \ddot{\gamma} = -T_1 \sin{\alpha} \frac{d}{2} + T_2 \sin{\beta} \frac{d}{2} &=
    -\frac{mgd}{2} (\alpha - \beta) = 
    -\frac{mdx}{4} \Delta - \frac{md^2\gamma}{8} \Sigma
\end{split}
\end{equation}
where $I = ml^2 / 12$ is the moment of inertia of the rod
against the vertical axis through its center,
$\Sigma = g (l_1^{-1} + l_2^{-1})$ and $\Delta = g(l_1^{-1} - l_2^{-1})$.
Denoting $x = k d\gamma$ for a specific mode we get
\begin{equation}
\begin{split}
    k\ddot{\gamma} = -\frac{k \Sigma}{2} \gamma - \frac{\Delta}{4} \gamma\\
    \frac{\ddot{\gamma}}{12} = -\frac{k \Delta}{4} \gamma - \frac{\Sigma}{8} \gamma
\end{split}
\hspace{1cm} \implies \hspace{1cm}
\begin{split}
    \omega^2 &= -\frac{\Sigma}{2} - \frac{\Delta}{4k}\\
    \omega^2 &= -3 k \Delta - \frac{3 \Sigma}{2}
\end{split}
\end{equation}
After some trivial calculations one can get
\begin{equation}
    \omega_{2, 3} = \sqrt{\frac{g}{l_1} + \frac{g}{l_2}
    \pm g \sqrt{\frac{1}{l_1^2} + \frac{1}{l_2^2} - \frac{1}{l_1 l_2}}}
\end{equation}

Notice that two of the frequencies reduce to $\sqrt{g/l}$ in case $l_1 = l_2 = l$.
