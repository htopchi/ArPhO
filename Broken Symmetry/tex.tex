\addcontentsline{toc}{section}{1 + 1 (3)}
\section*{1 + 1}

\subsection*{Problem}
The resistance of a thermistor depends on its absolute temperature as
$R = \beta T^2$, where $\beta = 7.00 \cdot 10^{-4}\text{ Ohm/K}$.
Surface area of the thermistor is $S = 1.00 \cdot 10^{-2} \text{ m}^2$ and
the heat transfer coefficient with the medium is $\alpha = 200 \text{ W/m$^2$K}$.
The temperature of the medium is $T_0 = 293 \text{ K}$.

\begin{enumerate}
    \item Find the current in such a thermistor under voltage $U_0 = 1000 \text{ V}$.
    \item Find the current in a system with two consequently connected thermistors
        under voltage $U_0 = 1000 \text{ V}$.
\end{enumerate}

One could notice a strange phenomenon while solving the previous question.

\begin{enumerate}
    \setcounter{enumi}{2}
    \item Find the minimal voltage $U_m$ when the "strange behavior" occurs
    in the system of two consequently connected thermistors.
\end{enumerate}

\subsection*{Useful information}
\begin{itemize}
    \item The solution of equation $x^3 + k x^2 + b = 0$ is given by
        \begin{equation}
            x = \frac{1}{3} \inb[{
                \inb({\frac{r}{2}})^{1/3} +
                k^2 \inb({\frac{r}{2}})^{-1/3} - k}], \hspace{1cm}
            r = 3^{3/2} \sqrt{27 b^2 + 4 b k^3} - 27b - 2 k^3
        \end{equation}
    \item A solution of any equation can be found even if
        the equation can not be solved analytically.
\end{itemize}

\subsection*{Solution}

The heat generated under voltage $U$ in case of temperature $T$
is given by $Q_+ = U^2 / \beta T^2$.
The heat loss is given by $Q_- = \alpha S (T - T_0)$.
The equilibrium condition $Q_+ = Q_-$ gives
\begin{equation}
    T^3 - T_0 T^2 - \frac{U^2}{\alpha \beta S} = 0
    \labele{T}
\end{equation}
For voltage $U = U_0$ we will get
$T = 1000 \text{ K}$ and $I = U / \beta T^2 = 1.42 \text{ A}$.

For the case of two consequently connected thermistors
one might think that each of them will get voltage $U = U_0 / 2$
because of symmetry.
Then we calculate the temperature $T = 680 \text{ K}$ and
current $I = 1.55 \text{ A}$.
Immediately we see that there is some problem,
because we got higher current for lower voltage.
One may also check the current is locally decreasing at that point
(this can simply done numerically).

This means that the solution is not stable. Indeed.
Suppose the voltage slightly increases on the first thermistor
and thereby decreases on the second one.
This will result in current decrease on the first thermistor
and increase on the second one,
which in its turn will result in negative charge accumulation in the center
thereby further increasing the voltage deviation.

So we need to find another solution,
which obviously will be non-symmetric.
The equation to be solved is
\begin{equation}
    I(U) = I(U^*) \hspace{1cm} U^* = U_0 - U
    \labele{IUI}
\end{equation}
where $I(U)$ is $U / \beta T^2$ and $T$ is the solution of \refe{T}.

This is to be done numerically.
Remember that the current is decreasing on $U$ at $U = U_0 / 2$,
so for slightly less $U$-s $I(U) > I(U^*)$.
Now we need to find the $U$, where $I(U)$ becomes bigger than $I(U^*)$.
The calculations are
\begin{center}
    \begin{tabular}{ c | c | c | c | c | c}
    $U$ [V]  & $U^*$ [V] & $T_U$ [K] & $T_{U^*}$ [K] & $I_U$ [A] & $I_{U^*}$ [A] \\
    \hline
    0     & 1000 & 293 & 1003 & 0     & 1.420 \\
    500   & 500  & 680 & 680  & 1.546 & 1.546 \\
    250   & 750  & 484 & 850  & 1.526 & 1.484 \\
    150   & 850  & 396 & 913  & 1.369 & 1.458 \\
    \rowcolor{yellow}
    200   & 800  & 440 & 881  & 1.473 & 1.471 \\
    190   & 810  & 431 & 888  & 1.458 & 1.468 \\
    \rowcolor{yellow}
    195   & 805  & 436 & 885  & 1.466 & 1.470 \\
    \end{tabular}
\end{center}
At this point we are sure that the equilibrium current is $I = 1.47 \text{ A}$.

One can see, that $I(U)$ is an increasing and then decreasing function on $U$.
Therefore if the overall $U_0$ is chosen to be $U_0 > 2\arg\max(I(U))$,
there will be three solutions to \refe{IUI} and a single solution otherwise.
Thus $U_m = 2 \arg\max(I(U))$, which can be found numerically again.

\begin{center}
    \begin{tabular}{ c | c }
    $U$ [V] & $I_U$ [A] \\
    \hline
    250 & 1.526 \\
    500 & 1.546 \\
    400 & 1.561 \\
    450 & 1.555 \\
    350 & 1.560 \\
    375 & 1.561 \\
    385 & 1.561 \\
    410 & 1.560 \\
    360 & 1.561 \\
    \end{tabular}
\end{center}

It is clear that the precision of $U_m$ is low,
since $I_U$ is given with 3 significant digits.
So the answer will be $U_m = 750 \pm 50 \text{ V}$.
