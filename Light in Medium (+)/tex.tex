\addcontentsline{toc}{section}{Light in Medium (5)}
\section*{Light in Medium}

\subsection*{Problem}

\renewcommand{\labelenumii}{\theenumii}
\renewcommand{\theenumii}{\theenumi.\arabic{enumii}.}

\begin{enumerate}

{\bfseries
\item Dispersion
}

One can model the atom as a stationary object with a heavy nucleus
and a light electron with mass $m$ and charge $q < 0$.
We assume that electron's charge is uniformly distributed
in a ball of radius $R$ with its center being near the nucleus.

\begin{enumerate}
    \item Find the natural frequency $\omega_0$ of the atom's oscillations.
\end{enumerate}

\hspace{.5cm}
The emission and absorption of photons by the atom
is the likeliest near the natural frequency of the atom.
As long as we want the medium to be transparent for the light
we will be using frequencies far away form the natural frequency.

\hspace{.5cm}
An electromagnetic wave is called linearly polarized
if its electric field strength is given by
$\vec{E} = \vec{E}_0 \cos(\omega t - k z)$ with $\vec{E}_0 \perp \hat{z}$.

\begin{enumerate}
    \setcounter{enumii}{1}
    \item Find the low of motion of the atom
    in a linearly polarized light with amplitude $\vec{E}$.

    \item What is the strength of electric field inside a capacitor
    with dipole moment density $p$?

    \item What is the refractive index of the medium
    with atom concentration $\rho$ for light of frequency $\omega$?
\end{enumerate}

{\bfseries
\item Faraday effect
}

\hspace{.5cm}
An electromagnetic wave is called circularly polarized
if its electric field strength is given by
$E_x = E_0 \cos(\omega t - k z)$, $E_y = \pm E_0 \sin(\omega t - k z)$.

\begin{enumerate}
    \item How will such waves be written in complex representation,
    if we want the $x$ component to be the real part
    and the $y$ component to be the imaginary part?
    How is $k$ given in terms of refractive index of the medium
    and other known parameters?

    \item Find the low of motion of the atom
    described in previous section in such field.
\end{enumerate}

\hspace{.5cm}
Suppose the medium is placed in a magnetic field of strength $B$
along the direction of light propagation.

\begin{enumerate}
    \setcounter{enumii}{2}
    \item How will the magnetic field affect the motion of atoms,
    if we consider that effect to be very small?

    \item Find the refractive index of the medium with concentration $\rho$
    inside magnetic field $B$ for clockwise and counterclockwise polarized light.
\end{enumerate}

The linearly polarized wave can be represented as
a sum of two circularly polarized waves.

\begin{enumerate}
    \setcounter{enumii}{4}
    \item How will the linearly polarized light be different
    at distance $l$ in the medium?
\end{enumerate}

\end{enumerate}

\subsection*{Useful information}
\begin{itemize}
    \item The light speed in medium is given by
    $c = 1/\sqrt{\varepsilon_0 \mu_0 \varepsilon \mu}$.
\end{itemize}

\subsection*{Solution}

\begin{enumerate}

{\bfseries
\item Dispersion
}

\begin{enumerate}
    \item Atom's natural frequency
\end{enumerate}

\hspace{.5cm}
In the whole problem we will consider a static nucleus
and a moving electron cloud,
as the later's mass is much smaller

\hspace{.5cm}
The interaction force between the nucleus and the electron
is linearly proportional to the cloud's center's displacement $\vec{x}$,
as only the part of the cloud within radius $x$ matters.
The force on the electron is given by
\begin{equation}
    \vec{F} = -\frac{k q}{x^2} \cdot q\frac{x^3}{R^3} \cdot \frac{\vec{x}}{x} =
    -\frac{k q^2}{R^3} \cdot \vec{x}
\end{equation}
So the natural frequency will be $\omega_0 = \sqrt{k q^2 / m R^3}$

\begin{enumerate}
    \setcounter{enumii}{1}
    \item Stimulated motion of an atom
\end{enumerate}

\hspace{.5cm}
The motion equation inside an external electric field if given by
$m\ddot{\vec{x}} = -m \omega_0^2 x + E q$.
If $E = E_0 \cos(\omega t)$ at some given point,
we should look for $x$ in form $x = x_0 \cos(\omega t)$.
Simple calculations lead to result
\begin{equation}
    x_0 = \frac{E_0 q}{m (\omega_0^2 - \omega^2)}
    \labele{x_on_E}
\end{equation}

\hspace{.5cm}
It might seem unnatural that $x_0$ changes the sign
when $\omega$ passes $\omega_0$,
but one should remember, that everything we discuss
is correct only for $\omega$-s far form $\omega_0$.

\begin{enumerate}
    \setcounter{enumii}{2}
    \item Electric field in a capacitor
\end{enumerate}

\hspace{.5cm}
Consider a capacitor of area $S$ and thickness $d$,
that is charged with surface charge density $\pm \sigma$.
The overall dipole moment in this case is $P = \sigma S \cdot d$
(in direction towards the plate charged with $\sigma$)
and its density is thereby $\vec{p} = \sigma$.

\hspace{.5cm}
The electric field inside can be given by $E = \sigma / \varepsilon_0$
towards the plate with $-\sigma$.
So the field can be expressed in terms of $\vec{p}$ as
\begin{equation}    
    \vec{E} = -\vec{p} / \varepsilon_0
    \labele{E_on_p}
\end{equation}

\begin{enumerate}
    \setcounter{enumii}{3}
    \item Refractive index
\end{enumerate}

\hspace{.5cm}
The existence of a non-identity refractive index is a result
of the medium-created electric field,
which is a source of non-identity permittivity.

\hspace{.5cm}
Electric permittivity is the ratio of the initial external field
to what we have in the medium as a result of superposition with the medium's field.
The source of medium's field is the polarization of atoms
under the influence of both external and medium-generated fields.
We will assume medium's field to have the same $\omega$ as the external field.
If we denote the later by $\vec{I} = \vec{I}_0 \cos(\omega t)$,
using \refe{x_on_E} will produce the polarization of a single atom
\begin{equation}
    \vec{P} = ex = \frac{\inb({\vec{E} + \vec{I}}) q^2}{m (\omega_0^2 - \omega^2)}
    \labele{p_on_EI}
\end{equation}

We already know the strength of the field that this will create.
Plugging \refe{p_on_EI} into \refe{E_on_p} gives
\begin{equation}
    \vec{I} = -\frac{\rho \inb({\vec{E} + \vec{I}}) q^2}{m (\omega_0^2 - \omega^2) \varepsilon_0}
    \implies
    \varepsilon = \frac{\vec{E}}{\vec{E} + \vec{I}} =
    1 + \frac{\rho q^2}{m (\omega_0^2 - \omega^2) \varepsilon_0}
\end{equation}

\hspace{.5cm}
The corresponding refractive index is then just $n = \sqrt{\varepsilon}$.

{\bfseries
\item Faraday effect
}

\begin{enumerate}
    \item Complex representation
\end{enumerate}

\hspace{.5cm}
In the complex representation $\vec{E}$ is just written as
$E = E_x + i E_y = e^{\pm i (\omega t - kz)}$.
$k$ is found from condition $k \lambda = 2 \pi$ and the fact
that light travels distance $\lambda$ in time $n\lambda / c$.
So $k = n\omega / c$.

\begin{enumerate}
    \setcounter{enumii}{1}
    \item Stimulated motion of an atom
\end{enumerate}

\hspace{.5cm}
The motion equation is the same: $m \ddot{x} = -m \omega_0^2 + qE$.
The difference is, that we look for the solution in form $x = x_0 e^{\pm \omega t}$.
It is straightforward to see, that $x_0$ is that same as \refe{x_on_E}.

\begin{enumerate}
    \setcounter{enumii}{2}
    \item Motion with magnetic field
\end{enumerate}

\hspace{.5cm}
The magnetic field acts on the moving particles with force
$\vec{F} = q \vec{V} \times \vec{B}$.
If $V$ in in complex representation and $B$ is along $z$,
it is possible to see that
the force in complex representation
is given by $F = -iqVB = -iq \dot{x} B$.
This term should be added to the motion equation,
which then becomes $m \ddot{x} = -m \omega_0^2 + qE - iq\dot{x}B$.
The solution is then easily fount in the same form.
\begin{equation}
    x = x'_0 e^{\pm i (\omega t - kz)} \hspace{1cm}
    x'_0 = \frac{E_0 q}{m (\omega_0^2 - \omega^2) \mp q B\omega} \approx
    \frac{E_0 q}{m (\omega_0^2 - \omega^2)}
    \inb({1 \pm \frac{qB\omega}{m (\omega_0^2 - \omega^2)} })
\end{equation}

\begin{enumerate}
    \setcounter{enumii}{3}
    \item Refractive indices
\end{enumerate}

\hspace{.5cm}
The refractive indices are found just like in 1.4.
The permittivities for the two cases are
\begin{equation}
    \varepsilon_{\pm} = 1+
    \frac{\rho q^2}{m (\omega_0^2 - \omega^2) \varepsilon_0}
    \inb({1 \pm \frac{qB\omega}{m (\omega_0^2 - \omega^2)} })
\end{equation}
And the refractive indices are
\begin{equation}
    n_{\pm} = \sqrt{\varepsilon_{\pm}} \approx n_0 \pm \Delta n
    \hspace{1cm}
    n_0 = \sqrt{1 + \frac{\rho q^2}{m (\omega_0^2 - \omega^2) \varepsilon_0}}
    \hspace{1cm}
    \Delta n = \frac{1}{n_0} \frac{\rho q^3 B\omega}{2 m^2 (\omega_0^2 - \omega^2)^2 \varepsilon_0}
\end{equation}
The approximation here is a simple Taylor expansion.

\begin{enumerate}
    \setcounter{enumii}{4}
    \item Linearly polarized light
\end{enumerate}

\hspace{.5cm}
The decomposition into two circularly polarized waves is simply written as
\begin{equation}
    E = E_0 e^{i (\omega t - k_+ z)} + E_0 e^{-i (\omega t - k_- z)}
\end{equation}
Outside of the medium, where $k_+ = k_-$,
this is indeed a linearly polarized light with $E = 2E_0 \cos(\omega t - k z)$.
Inside the medium $k_{\pm} = (n_0 \pm \Delta n) \omega / c = k_0 \pm \Delta k$.
The expression for the field then can be rewritten as
\begin{equation}
    E = E_0 e^{i (\omega t - k_0 z)} e^{-i \Delta k z} +
        E_0 e^{-i (\omega t - k_0 z)} e^{-i \Delta k z} 
    = E_{B=0} e^{-i \Delta k z}
\end{equation}

\hspace{.5cm}
Here we have an additional rotation of plane of $E$
compared to the case without magnetic field.
As $q < 0$, $\Delta k < 0$.
This means that the rotation is clockwise.
After distance $l$ the plane will have been rotated by angle
$\beta = l \Delta n \omega / c$ counterclockwise
($\beta$ itself is negative).

\end{enumerate}