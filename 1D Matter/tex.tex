\addcontentsline{toc}{section}{One-Dimensional Matter (5)}
\section*{One-Dimensional Matter}

\subsection*{Problem}

\renewcommand{\labelenumii}{\theenumii}
\renewcommand{\theenumii}{\theenumi.\arabic{enumii}.}

\begin{enumerate}
    \item What dipole moment will a conducting ball of radius $R$ gain
        in an electric field of strength $E$?
\end{enumerate}

Suppose a system that consists of infinitely many balls of radius $R$
placed along a single line with distance $l \gg R$ between each other.
The balls are charged $\pm Q$, with any neighboring charges being opposite.
The whole system is placed inside an electric filed $E$
perpendicular to the axis of alignment.

\begin{enumerate}
    \setcounter{enumi}{1}
    \item Find the interaction energy of one ball with all the others.
    Charge induction because of the interaction can be neglected.
\end{enumerate}

Suppose all the balls are put on a single thin non-conducting rod,
so they can't move in any other direction then the rod itself.
No friction is involved.

\begin{enumerate}
    \setcounter{enumi}{2}
    \item Calculate the equilibrium distance $l_0$ between the balls.
    \item Determine the longitudinal wave propagation speed in this system,
    knowing that in ordinary matter the speed of sound is given by $c = \sqrt{Y/\rho_0}$,
    where $Y$ is the Young's modulus and $\rho_0$ is the density.
    All balls have density $\rho$.
\end{enumerate}

Suppose the radii of the balls are not equal,
but it is not considerable for nearby balls.
In other words the ball radius slowly changes
when moving along the the arrangement axis.
Of course a radius of a single ball remains constant.
Let's also say the electric field changes over time,
but it's always uniform.

At some time point $t_c$
one creates a mechanical signal of duration $\Delta t_c$ inside the system.
$\Delta t_c$ is much bigger then the characteristic time of $E$-s change.
Some long time later the signal is observed and it has duration $\Delta t_0$.
Assume that the distance between the observation and creation points
is much larger then the length of the signal,
which itself is much bigger than the distance between the balls.

\begin{enumerate}
    \setcounter{enumi}{4}
    \item Find the strength of the electric field at time point $t_c$,
    if it was $E_0$ when the signal was observed.
\end{enumerate}

\subsection*{Useful information}
\begin{itemize}
\item Some expressions of knows sums
\begin{equation}
    \sum_{k=1}^{\infty} \frac{(-1)^k}{k} = -\ln{2}
    \hspace{1cm}
    \sum_{k=1}^{\infty} \frac{1}{k^2} = \frac{\pi^2}{6}
    \hspace{1cm}
    \sum_{k=1}^{\infty} \frac{1}{k^3} = \zeta
\end{equation}

\item Using the ball index as a coordinate might be a good idea.

\end{itemize}

\subsection*{Solution}

\begin{enumerate}
    \item Dipole moment of a ball in external field
\end{enumerate}

This is a famous problem.
The only reason of including it here is that the result is needed later.
If we consider two uniformly charged balls of radius $R$ and charges $\pm q$,
with displacement $\vec{d} \ll R$,
the result will be a non-uniformly charged sphere of radius $R$.

To calculate the field inside this object
we have to sum up the fields of the two balls.
It is known, that at some distance $r$ form the center
only the charge "inside" that distance crates field.
Suppose the ball with charge $-q$ is at $0-0$
and the other one is at $\vec{d}$.
The dipole moment is $\vec{P} = q \vec{d}$.
Their field is then given by
\begin{equation}
    E(\vec{r}) = \frac{-k q}{r^2} \cdot \frac{r^3}{R^3} \cdot \frac{\vec{r}}{r} +
        \frac{k q}{r_d^2} \cdot \frac{r_d^3}{R^3} \cdot \frac{\vec{r_d}}{r_d} =
        -\frac{kq}{R^3}(\vec{r} - \vec{r}_d) = -\frac{k \vec{P}}{R^3}
\end{equation}
where $\vec{r}_d = \vec{r} - \vec{d}$.

As the field inside is uniform,
once inside an external electric field,
the ball will charge in way we just discussed
to compensate the external field inside of it
(the field in a conductor should be $0$).
This will be achieved if the ball gains dipole moment
$\vec{P} = \vec{E} R^3 / k$.

\begin{enumerate}
    \setcounter{enumi}{1}
    \item Interaction energy
\end{enumerate}

The interaction energy consists of two components:
charge-charge interactions and dipole-dipole interactions.
Any charge-dipole interaction in the discussed system is $0$,
as the "charges" of dipoles are equally displaced in direction
perpendicular to the one connecting them to the single charge.

The sum of charge interaction energies is simply given by
\begin{equation}
    \frac{W_{cc}}{2} = -\frac{kQ^2}{l} + \frac{kQ^2}{2l} - \frac{kQ^2}{3l} + \cdots =
    -\frac{kQ^2}{l} \ln{2}
\end{equation}
where the overall $1/2$ makes sure we sum up over the balls
both to the right and to the left of the considered one.

To calculate a single dipole-dipole interaction energy
let's depict each dipole as charges $\pm q$ with displacement $d$.
The dipoles are aligned, and their distance is $r$.
The energy is then given by
\begin{equation}
    W^{(s)}_{dd} = \frac{2kq^2}{r} - \frac{2kq^2}{\sqrt{r^2 + d^2}} =
    \frac{2 k q^2}{r} \inb({1 - \frac{1}{\sqrt{1+(d/r)^2}}}) =
    \frac{k q^2 d^2}{r^3}
\end{equation}

Summation by $r \in \{l, 2l, 3l, ...\}$ (each twice) produces
\begin{equation}
    W_{dd} = \frac{2 k P^2}{l^3} \zeta
\end{equation}
where $P$ is the dipole moment $qd$.
The overall energy in then just $W = W_{cc} + W_{dd}$.

\begin{enumerate}
    \setcounter{enumi}{2}
    \item Equilibrium distance
\end{enumerate}

The condition for an equilibrium is the condition of minimal energy.
\begin{equation}
    \derivative{W}{l} = \frac{k Q^2}{l^2} 2\ln{2} - \frac{k P^2}{l^4} 6\zeta
\end{equation}
and after plugging in the expression for $P$,
the condition $\text{d}W/\text{d}l = 0$ leads to
\begin{equation}
    l_0 = \sqrt{\frac{3\zeta}{\ln{2}}} \cdot \frac{E R^3}{k Q}
\end{equation}

\begin{enumerate}
    \setcounter{enumi}{3}
    \item Speed of sound
\end{enumerate}

Firstly we have to calculate the one-dimensional analogues of $Y$ and $\rho_0$.
The situation with $\rho_0$ is really simple.
It should be replaced with linear density $\lambda$,
which is simply given by $4\pi R^3 \rho / 3 l_0$.

The analogue of $Y$ is something like a spring strength.
It's the ratio of tension (instead of pressure for 3D case)
which occurs in case of deformation $\varepsilon = \Delta l / l_0$
to the deformation itself.
This can be calculated as
\begin{equation}
    \kappa = \frac{T}{\varepsilon} = -l_0 \inb.{\derivative[2]{W}{l}}|_{l=l_0} =
    \frac{k Q^2}{l_0^2} 4\ln{2} - \frac{k P^2}{l_0^4} 24\zeta =
    \frac{4 \ln^2{2}}{3\zeta} \frac{k^3 Q^4}{E^2 R^6}
\end{equation}
and for the speed of wave we get
\begin{equation}
    c = \sqrt{\frac{\kappa}{\lambda}} =
    \frac{(\ln{2})^{3/4} 3^{1/4}}{\zeta^{1/4} \pi^{1/2}}
    \cdot \frac{k Q^{3/2}}{\rho^{1/2}} \cdot \frac{E^{3/2}}{R^3}
\end{equation} 

\begin{enumerate}
    \setcounter{enumi}{4}
    \item Signal in variating medium
\end{enumerate}

We have to write the equations for the head and the tail of the signal
and try to understand what happens to it.
For both of them we know $c(x, t) \differential{t} = \differential{x}$.
It would be great if we could factorize $c(x, t)$ into $f(x)g(t)$. 

As $E$ changes over time,
the dependency $R(x)$ will also change.
A more "stable" coordinate in that sense is the ball index $n$.
This way $R$ depends only on $n$.
The wave speed in those coordinates is given by
$c_n = \text{d}n / \text{d}t = c / l_0 = \beta E^{1/2} R^{-6}$,
where $\beta$ is some unimportant constant, as we will soon see.

The propagation equation is written as
$c_n(n, t) \differential{t} = \differential{n}
\iff \beta E^{1/2}(t) \differential{t} = R^6(n) \differential{n}$.
If we integrate both sides of the equation
from the creation to observation for both the head and the tail, we get
\begin{equation}
    \beta\int\displaylimits_{t=t_c}^{t_0} E^{1/2}(t) \differential{t} =
    \int\displaylimits_{n=n_c}^{n_0} R^6(n) \differential{n}
    \hspace{1cm}
    \beta\int\displaylimits_{t=t_c+\Delta t_c}^{t_0 + \Delta t_0} E^{1/2}(t) \differential{t} =
    \int\displaylimits_{n=n_c}^{n_0} R^6(n) \differential{n}
\end{equation}
By equalizing the two and dropping the common parts
of the leftover integrals we simply get
$\beta E^{1/2}(t_c) \Delta t_c = \beta E^{1/2}(t_0) \Delta t_0$,
which implies $E_c = E_0 (\Delta t_0 / \Delta t_c)^2$.
